%%%%%%%%%%%%%%%%%%%%%%%%%%%%%%%%%%%%%%%%%
% Beamer Presentation
% LaTeX Template
% Version 1.0 (10/11/12)
%
% This template has been downloaded from:
% http://www.LaTeXTemplates.com
%
% License:
% CC BY-NC-SA 3.0 (http://creativecommons.org/licenses/by-nc-sa/3.0/)
%
%%%%%%%%%%%%%%%%%%%%%%%%%%%%%%%%%%%%%%%%%

%----------------------------------------------------------------------------------------
%	PACKAGES AND THEMES
%----------------------------------------------------------------------------------------

\documentclass[UTF8,aspectratio=169]{ctexbeamer}

\usepackage{hyperref}
\hypersetup{
	colorlinks=true,
	linkcolor=red,
	anchorcolor=blue,
	citecolor=green
}

\mode<presentation> {
	
	% The Beamer class comes with a number of default slide themes
	% which change the colors and layouts of slides. Below this is a list
	% of all the themes, uncomment each in turn to see what they look like.
	
	%\usetheme{default}
	%\usetheme{AnnArbor}
	%\usetheme{Antibes}
	%\usetheme{Bergen}
	%\usetheme{Berkeley}
	%\usetheme{Berlin}
	%\usetheme{Boadilla}
	%\usetheme{CambridgeUS}
	%\usetheme{Copenhagen}
	%\usetheme{Darmstadt}
	%\usetheme{Dresden}
	%\usetheme{Frankfurt}
	%\usetheme{Goettingen}
	%\usetheme{Hannover}
	%\usetheme{Ilmenau}
	%\usetheme{JuanLesPins}
	%\usetheme{Luebeck}
	\usetheme{Madrid}
	%\usetheme{Malmoe}
	%\usetheme{Marburg}
	%\usetheme{Montpellier}
	%\usetheme{PaloAlto}
	%\usetheme{Pittsburgh}
	%\usetheme{Rochester}
	%\usetheme{Singapore}
	%\usetheme{Szeged}
	%\usetheme{Warsaw}
	
	% As well as themes, the Beamer class has a number of color themes
	% for any slide theme. Uncomment each of these in turn to see how it
	% changes the colors of your current slide theme.
	
	%\usecolortheme{albatross}
	%\usecolortheme{beaver}
	%\usecolortheme{beetle}
	%\usecolortheme{crane}
	%\usecolortheme{dolphin}
	%\usecolortheme{dove}
	%\usecolortheme{fly}
	%\usecolortheme{lily}
	%\usecolortheme{orchid}
	%\usecolortheme{rose}
	%\usecolortheme{seagull}
	%\usecolortheme{seahorse}
	%\usecolortheme{whale}
	%\usecolortheme{wolverine}
	
	%\setbeamertemplate{footline} % To remove the footer line in all slides uncomment this line
	%\setbeamertemplate{footline}[page number] % To replace the footer line in all slides with a simple slide count uncomment this line
	
	%\setbeamertemplate{navigation symbols}{} % To remove the navigation symbols from the bottom of all slides uncomment this line
}

\usepackage{graphicx} % Allows including images
\usepackage{booktabs} % Allows the use of \toprule, \midrule and \bottomrule in tables

% Fonts
% \usepackage{libertine}
% \setmonofont{Courier}
\setCJKsansfont[ItalicFont=Noto Serif CJK SC Black, BoldFont=Noto Sans CJK SC Black]{Noto Sans CJK SC}

%----------------------------------------------------------------------------------------
%	TITLE PAGE
%----------------------------------------------------------------------------------------

\title[第1讲]{第1讲 :操作系统概述} % The short title appears at the bottom of every slide, the full title is only on the title page
\subtitle{第一节:课程概述}
\author{向勇、陈渝} % Your name
\institute[清华大学] % Your institution as it will appear on the bottom of every slide, may be shorthand to save space
{
	清华大学计算机系 \\ % Your institution for the title page
	\medskip
	\textit{xyong,yuchen@tsinghua.edu.cn} % Your email address
}
\date{\today} % Date, can be changed to a custom date

%\def\imagepath{./resources/graphics}
%\usepackage[imagepath=\imagepath]{ditaa}
%\graphicspath{ {\imagepath/} }


%\usepackage{pgfpages}
%\setbeameroption{show notes on second screen}
%%----------------------------------------------------------------------------------------
%	TITLE PAGE
%----------------------------------------------------------------------------------------

\title[第9讲]{第九讲 :进程和线程} % The short title appears at the bottom of every slide, the full title is only on the title page
\subtitle{第9节:进程地址空间与熔断(meltdown)漏洞}
\author{向勇、陈渝} % Your name
\institute[清华大学] % Your institution as it will appear on the bottom of every slide, may be shorthand to save space
{
	清华大学计算机系 \\ % Your institution for the title page
	\medskip
	\textit{xyong,yuchen@tsinghua.edu.cn} % Your email address
}
\date{\today} % Date, can be changed to a custom date


\begin{document}

\begin{frame}
\titlepage % Print the title page as the first slide
\end{frame}

%----------------------------------------------
\begin{frame}
\frametitle{提纲} % Table of contents slide, comment this block out to remove it
\tableofcontents % Throughout your presentation, if you choose to use \section{} and \subsection{} commands, these will automatically be printed on this slide as an overview of your presentation
\end{frame}
%----------------------------------------------
%%	PRESENTATION SLIDES
%----------------------------------------------
%------------------------------------------------
\section{第9节:进程地址空间与熔断(meltdown)漏洞}% Sections can be created in order to organize your presentation into discrete blocks, all sections and subsections are automatically printed in the table of contents as an overview of the talk
%------------------------------------------------
\subsection{背景知识回顾} % A subsection can be created just before a set of slides with a common theme to further break down your presentation into chunks
%------------------------------------------------
\begin{frame}[fragile]
    \frametitle{侧信道问题}
    \begin{block}{}
    假设有abc三个地址,其中a地址没有访问的权限,但是b和c可以访问,此时执行下面这个条件表达式:
    \begin{lstlisting}[language = C]
        x = a ?b:c
    \end{lstlisting}
    \end{block}
    \begin{block}{分析}
	    \begin{itemize}
	        \item a地址无法访问,系统会直接报错!
	        \item 采用多流水线的CPU在检查a的访问权限时,预先往下执行。
	        \item 权限检查完成时,已经根据a的结果完成了b或者c的加载到高速缓存,只还没有赋给x。
	    \end{itemize}
    \end{block}
\end{frame}
%------------------------------------------------
\begin{frame}[fragile]
    \frametitle{侧信道问题的影响}
	    \begin{itemize}
	        \item 虽然报错,但再次访问时会比较快;
	        \item 再次访问b和c,根据返回的时间快慢,可猜到a的值;
	        \item 这个问题导致了2018年元旦前后熔断漏洞(CVE-2017-5754:meltdown);
	        \item 操作系统需要来补救这个CPU设计问题:KAISER
	        \item 类似问题不止这一个......
	    \end{itemize}
\end{frame}
%------------------------------------------------

% #### 背景
% 
% 假设有abc三个地址,其中a地址没有访问的权限,但是b和c可以访问,此时执行下面这个条件表达式:
% 
% x= a?b:c
% 
% 表面看来,由于a地址无法访问,所以系统会直接报错!
% 但实际上,CPU为了加快执行速度,会采用多流水线作业方式。它会在检查a是否可访问的同时,预先就往下执行了。
% 等到权限检查结果回来,已经根据a的结果完成了b或者c的加载,只是还没有赋给x而已。经过加载的b或者c会在缓存里。虽然报错了,但如果再次访问就会比较快。
% 于是再次访问b和c,根据返回的时间快慢,就可以猜到a的内容!
% 
%------------------------------------------------
\begin{frame}
    \frametitle{CPU高速缓存结构(Intel Skylake)}
	  %% figure
	  	\begin{figure}
	  	\includegraphics[width=0.9\linewidth]{figs/cache-in-skylake.png}
	%  	\caption{xxxx}
	  	\end{figure}
\end{frame}
%------------------------------------------------
\begin{frame}
    \frametitle{各级存储结构的访问延迟}
\begin{table}[]
\begin{tabular}{|l|l|}
\cline{1-2}
访问类型        & 延迟        \\ \cline{1-2} 
L1 cache命中  & 约4个时钟周期  \\ \cline{1-2}
L2 cache 命中 & 约10个时钟周期 \\ \cline{1-2}
L3 cache命中  & 约40个时钟周期 \\ \cline{1-2}
访问本地DDR & 约60 纳秒 \\ \cline{1-2}
访问远端内存节点DDR & 约100 纳秒  \\ \cline{1-2}
\end{tabular}
\end{table}

\end{frame}
%------------------------------------------------
% #### CPU高速缓存结构(Intel Skylake)
% 
% [How Does CPU Cache Work?](https://www.makeuseof.com/tag/what-is-cpu-cache/)
% 
% ![cache-in-skylake](figs/cache-in-skylake.png)
% 
% https://mp.weixin.qq.com/s/zlspXeDGlAEzVsq2h6gg8w
% 
% 各级存储结构的访问延迟
% 
% | 访问类型            | 延迟           |
% | ------------------- | -------------- |
% | L1 cache命中        | 约4个时钟周期  |
% | L2 cache 命中       | 约10个时钟周期 |
% | L3 cache命中        | 约40个时钟周期 |
% | 访问本地DDR         | 约60 纳秒      |
% | 访问远端内存节点DDR | 约100纳秒      |
% 
%------------------------------------------------
\begin{frame}
  %% columns
      \begin{columns}
      \begin{column}{0.4\textwidth}
	    \frametitle{指令执行的乱序优化(Intel Skylake)}
		  %% figure
		  	\begin{figure}
		  	\includegraphics[width=0.75\linewidth]{figs/skylake-out-of-order.png}
		%  	\caption{xxxx}
		  	\end{figure}
      \end{column}
      \begin{column}{0.6\textwidth}
		   \begin{block}{乱序执行过程}
		    \begin{itemize}
		        \item 获取指令和解码:保放到执行缓冲区
		        \item 乱序执行指令:保存在结果序列中
		        \item 退休期Retired Circle:重新排列结果序列及安全检查(如地址访问的权限检查),提交结果到寄存器
		    \end{itemize}
		    \end{block}
      \end{column}
      \end{columns}
\end{frame}
%------------------------------------------------
% #### 指令执行的乱序优化(Intel Skylake)
% https://meltdownattack.com/meltdown.pdf
% Meltdown: Reading Kernel Memory from User Space
% Figure 1: Simplified illustration of a single core of the Intel’s Skylake microarchitecture
% 
% ![skylake-out-of-order](figs/skylake-out-of-order.png)
% 
% https://www.zhihu.com/question/265012502?utm_medium=social&utm_source=wechat_session&from=groupmessage&isappinstalled=0
% 
% 乱序执行可以简单的分为三个阶段
% 
% 1. 获取指令,解码后存放到执行缓冲区Reservations Stations
% 2. 乱序执行指令,结果保存在一个结果序列中
% 3. 退休期Retired Circle,重新排列结果序列及安全检查(如地址访问的权限检查),提交结果到寄存器
% 
%------------------------------------------------
\begin{frame}
    \frametitle{CPU异常指令执行}
  %% figure
  	\begin{figure}
  	\includegraphics[width=0.8\linewidth]{figs/exception.png}
%  	\caption{xxxx}
  	\end{figure}

\end{frame}
%------------------------------------------------
% #### CPU异常指令执行
% 
% https://www.freebuf.com/vuls/159269.html
% CPU异常指令执行
% 
% ![exception](figs/exception.png)
% 
%------------------------------------------------
\begin{frame}
    \frametitle{CPU数据访问权限和地址合法性检查}
  %% figure
  	\begin{figure}
  	\includegraphics[width=0.8\linewidth]{figs/right-check.png}
%  	\caption{xxxx}
  	\end{figure}

\end{frame}
%------------------------------------------------
% #### CPU数据访问权限和地址合法性检查
% 
% https://www.freebuf.com/vuls/159269.html
% 图3 CPU数据访问权限和地址合法性检查
% 
% ![right-check](figs/right-check.png)
% 
%------------------------------------------------
\subsection{熔断漏洞} % A subsection can be created just before a set of slides with a common theme to further break down your presentation into chunks
%------------------------------------------------
\begin{frame}
    \frametitle{xx}
    \begin{itemize}
        \item xx
    \end{itemize}
\end{frame}
%------------------------------------------------

% #### 熔断漏洞(CVE-2017-5754):利用过程
% 
% https://meltdownattack.com/meltdown.pdf
% Meltdown: Reading Kernel Memory from User Space
% 
% ![meltdown-poc](figs/meltdown-poc.jpg)
% 
%  1. 接收者开辟一段2^8=256个page大小的内存(256*4096)作为probe array,并保证这部分内存未被缓存。
%  2. 假设要访问的非法内存的地址存在rcx, % 通过mov指令读取位于rcx地址的内存中的一个字节,存在rax中,这条指令将来会产生异常,然而在异常产生前,就可以通过transient % instructions把读取的内容发送出去:
%  3. 假设读取的值是i,那么就在transient instructions里面访问probe array的第i*4096个元素,这会导致第i*4096个元素被缓存。
%  4. 接收者通过测量所有的256个page的内存的访问时间,就可以知道i的值了。
% 
% #### 熔断漏洞:原理
% https://mp.weixin.qq.com/s/2FvvFUT8taRPv6GOHzNW-g
% 图4 漏洞原理图
% 
% ![meltdown-method](figs/meltdown-method.jpg)
% 
% #### 熔断漏洞:在用户态读取内核数据
% 
% https://meltdownattack.com/meltdown.pdf
% Meltdown: Reading Kernel Memory from User Space
% Figure 4
% 
% ![access-duration](figs/access-duration.png)
% 
%------------------------------------------------
\subsection{进程用户态和内核态的隔离} % A subsection can be created just before a set of slides with a common theme to further break down your presentation into chunks
%------------------------------------------------
\begin{frame}
    \frametitle{xx}
    \begin{itemize}
        \item xx
    \end{itemize}
\end{frame}
%------------------------------------------------
% #### KPTI: Kernel page-table isolation
% https://gruss.cc/files/kaiser.pdf
% KASLR is Dead: Long Live KASLR
% Fig. 2
% 
% ![kaiser](figs/kaiser.png)
% 
% (a) The kernel is mapped into the address space of every user process.
% (b) Theoretical concept of stronger kernel isolation. It splits the address spacesand  only  interrupt  handling  code  is  % mapped  in  both  address  spaces. 
% (c)  Forcompatibility with x86 Linux, KAISER relies on SMAP to prevent invalid usermemory references and SMEP to prevent % execution of user code in kernel mode.
% 
% supervisor-mode access prevention (SMAP) and supervisor-mode execution prevention (SMEP)
% 
% #### Shadow address space in KAISER
% https://gruss.cc/files/kaiser.pdf
% KASLR is Dead: Long Live KASLR
% Fig. 3: Shadow address space
% 
% Kernel Address Isolation to have Side channels Efficiently Removed, KAISER
% 
% ![kaiser-shadow-AS](figs/kaiser-shadow-AS.png)
% 
% PML4 of user address space and kernel addressspace are placed next to each other in physical memory.
% 
% https://en.wikipedia.org/wiki/Kernel_page-table_isolation
% 
% 
% #### “骑士” 漏洞(CVE-2019-11157)
% 
% http://voltjockey.com/flies/paper/2.pdf
% VoltJockey: Breaking SGX by Software-ControlledVoltage-Induced Hardware Faults
% Fig. 1.   Overview of our voltage-induced fault attack
% 
% ![VoltJockey](figs/VoltJockey.png)
% 
% 动态电源管理模块DVFS(Dynamic Voltage and Frequency Scaling)允许多核处理器根据负载信息采用相应的频率和电压运行,以降低处理器的功耗。
% 当一个核出现电压和频率不太匹配(如电压偏低无法满足较高频率运行需求)时,系统就会出现短暂“故障”。
% 故障对系统行为结果的干扰会泄露出的系统行为信息。
%----------------------------------------------

\end{document}
