%%%%%%%%%%%%%%%%%%%%%%%%%%%%%%%%%%%%%%%%%
% Beamer Presentation
% LaTeX Template
% Version 1.0 (10/11/12)
%
% This template has been downloaded from:
% http://www.LaTeXTemplates.com
%
% License:
% CC BY-NC-SA 3.0 (http://creativecommons.org/licenses/by-nc-sa/3.0/)
%
%%%%%%%%%%%%%%%%%%%%%%%%%%%%%%%%%%%%%%%%%

%----------------------------------------------------------------------------------------
%	PACKAGES AND THEMES
%----------------------------------------------------------------------------------------

\documentclass[UTF8,aspectratio=169]{ctexbeamer}

\usepackage{hyperref}
\hypersetup{
	colorlinks=true,
	linkcolor=red,
	anchorcolor=blue,
	citecolor=green
}

\mode<presentation> {
	
	% The Beamer class comes with a number of default slide themes
	% which change the colors and layouts of slides. Below this is a list
	% of all the themes, uncomment each in turn to see what they look like.
	
	%\usetheme{default}
	%\usetheme{AnnArbor}
	%\usetheme{Antibes}
	%\usetheme{Bergen}
	%\usetheme{Berkeley}
	%\usetheme{Berlin}
	%\usetheme{Boadilla}
	%\usetheme{CambridgeUS}
	%\usetheme{Copenhagen}
	%\usetheme{Darmstadt}
	%\usetheme{Dresden}
	%\usetheme{Frankfurt}
	%\usetheme{Goettingen}
	%\usetheme{Hannover}
	%\usetheme{Ilmenau}
	%\usetheme{JuanLesPins}
	%\usetheme{Luebeck}
	\usetheme{Madrid}
	%\usetheme{Malmoe}
	%\usetheme{Marburg}
	%\usetheme{Montpellier}
	%\usetheme{PaloAlto}
	%\usetheme{Pittsburgh}
	%\usetheme{Rochester}
	%\usetheme{Singapore}
	%\usetheme{Szeged}
	%\usetheme{Warsaw}
	
	% As well as themes, the Beamer class has a number of color themes
	% for any slide theme. Uncomment each of these in turn to see how it
	% changes the colors of your current slide theme.
	
	%\usecolortheme{albatross}
	%\usecolortheme{beaver}
	%\usecolortheme{beetle}
	%\usecolortheme{crane}
	%\usecolortheme{dolphin}
	%\usecolortheme{dove}
	%\usecolortheme{fly}
	%\usecolortheme{lily}
	%\usecolortheme{orchid}
	%\usecolortheme{rose}
	%\usecolortheme{seagull}
	%\usecolortheme{seahorse}
	%\usecolortheme{whale}
	%\usecolortheme{wolverine}
	
	%\setbeamertemplate{footline} % To remove the footer line in all slides uncomment this line
	%\setbeamertemplate{footline}[page number] % To replace the footer line in all slides with a simple slide count uncomment this line
	
	%\setbeamertemplate{navigation symbols}{} % To remove the navigation symbols from the bottom of all slides uncomment this line
}

\usepackage{graphicx} % Allows including images
\usepackage{booktabs} % Allows the use of \toprule, \midrule and \bottomrule in tables

% Fonts
% \usepackage{libertine}
% \setmonofont{Courier}
\setCJKsansfont[ItalicFont=Noto Serif CJK SC Black, BoldFont=Noto Sans CJK SC Black]{Noto Sans CJK SC}

%----------------------------------------------------------------------------------------
%	TITLE PAGE
%----------------------------------------------------------------------------------------

\title[第1讲]{第1讲 :操作系统概述} % The short title appears at the bottom of every slide, the full title is only on the title page
\subtitle{第一节:课程概述}
\author{向勇、陈渝} % Your name
\institute[清华大学] % Your institution as it will appear on the bottom of every slide, may be shorthand to save space
{
	清华大学计算机系 \\ % Your institution for the title page
	\medskip
	\textit{xyong,yuchen@tsinghua.edu.cn} % Your email address
}
\date{\today} % Date, can be changed to a custom date

\usepackage{tcolorbox}
%----------------------------------------------------------------------------------------
% TITLE PAGE
%----------------------------------------------------------------------------------------

\title[第21讲]{第二十一讲 :分布式系统} % The short title appears at the bottom of every slide, the full title is only on the title page
\subtitle{第2节:分布式文件系统}
\author{向勇、陈渝、李国良} % Your name
\institute[清华大学] % Your institution as it will appear on the bottom of every slide, may be shorthand to save space
{
    清华大学计算机系 \\ % Your institution for the title page
    \medskip
    \textit{xyong,yuchen,liguoliang@tsinghua.edu.cn} % Your email address
}
\date{\today} % Date, can be changed to a custom date

\begin{document}
    
    \begin{frame}
        \titlepage % Print the title page as the first slide
    \end{frame}
    
    %----------------------------------------------
    \begin{frame}
        \frametitle{提纲} % Table of contents slide, comment this block out to remove it
        \tableofcontents % Throughout your presentation, if you choose to use \section{} and \subsection{} commands, these will automatically be printed on this slide as an overview of your presentation
        
        %% itemize
        %Ref:
        %    \begin{itemize}
        %        \item \href{http://osq.cs.berkeley.edu/public/JFoster-Drivers.ppt}{Linux Device Drivers Overview}
        %        \item \href{http://ermak.cs.nstu.ru/understanding.linux.kernel.pdf}{Understanding the Linux Kernel}
        %    \end{itemize}
        
    \end{frame}
    %----------------------------------------------
    %%  PRESENTATION SLIDES
    %----------------------------------------------
    \section{第2节:分布式文件系统} % Sections can be created in order to organize your presentation into discrete blocks, all sections and subsections are automatically printed in the table of contents as an overview of the talk
    %----------------------------------------------
    \subsection{Sun的网络文件系统(NFS)} % A subsection can be created just before a set of slides with a common theme to further break down your presentation into chunks
    %----------------------------------------------
    \begin{frame}[fragile]
        \frametitle{分布式文件系统}
        \begin{itemize}
            \item 分布式文件系统是分布式客户端/服务器计算模型的早期应用
            \item 对于客户端应用程序,分布式文件系统似乎与本地文件系统没有任何不同
        \end{itemize} \pause
        
        \begin{tcolorbox}
        关键问题:{\color{blue}如何构建分布式文件系统?}
        \end{tcolorbox}
        
        
        %    \framesubtitle{xxxx}
        %% figure
            \begin{figure}
            \includegraphics[width=0.9\linewidth]{figs/distributed-fs.png}
          %  \caption{xxxx}
            \end{figure}
\end{frame}
%----------------------------------------------
\begin{frame}[fragile]
    \frametitle{网络文件系统(NFS, Network File System)}
    %    \framesubtitle{xxxx}
    由Sun Microsystems开发的最早且相当成功的分布式系统之一
    \pause
    \begin{itemize}
        \item 在NFS中,Sun只指定了客户端和服务器用于通信的确切消息格式
        \item 一种开放协议(open protocol),而不是构建专有的封闭系统
    \end{itemize}

        \begin{figure}
            \includegraphics[width=0.5\linewidth]{figs/sun-nfs.png}
            %  \caption{xxxx}
        \end{figure}
\end{frame}
%----------------------------------------------
\begin{frame}
    \frametitle{提纲} % Table of contents slide, comment this block out to remove it
    \tableofcontents % Throughout your presentation, if you choose to use \section{} and \subsection{} commands, these will automatically be printed on this slide as an overview of your presentation
    
    %% itemize
    %Ref:
    %    \begin{itemize}
    %        \item \href{http://osq.cs.berkeley.edu/public/JFoster-Drivers.ppt}{Linux Device Drivers Overview}
    %        \item \href{http://ermak.cs.nstu.ru/understanding.linux.kernel.pdf}{Understanding the Linux Kernel}
    %    \end{itemize}
    
\end{frame}
%----------------------------------------------
    \subsection{NFSv2: 无状态文件协议}
%----------------------------------------------
\begin{frame}[fragile]
    \frametitle{NFSv2协议}
%    \framesubtitle{xxxx}
关键技术
    \begin{itemize}
        \item 目标:简单快速的服务器崩溃恢复\pause
        \item 快速崩溃恢复的关键:无状态
        \item 服务器不会追踪客户正在做什么
    \end{itemize}
    \begin{figure}
    \includegraphics[width=0.65\linewidth]{figs/stateful-code.png}
    %  \caption{xxxx}
    \end{figure}
有状态(stateful)协议的示例。文件描述符是客户端和服务器之间的共享状态 
\end{frame}

%----------------------------------------------
\begin{frame}[fragile]
    \frametitle{NFSv2协议}
    %    \framesubtitle{xxxx}
    问题:共享状态使崩溃恢复变得复杂
    \begin{itemize}
        \item Srv:在第一次读取完成后,但在客户端发出第二次读取之前,服务器崩溃。
%        \item Srv:服务器启动并再次运行后,客户端会发出第二次读取,但服务器不知道fd指的是哪个文件?
%        \item Client:一个打开文件然后崩溃的客户端:open()在服务器上用掉了一个文件描述符,服务器如何关闭给定的文件呢?
    \end{itemize}
    \begin{figure}
        \includegraphics[width=1.0\linewidth]{figs/stateful-code.png}
        %  \caption{xxxx}
    \end{figure}

\end{frame}

%----------------------------------------------
\begin{frame}[fragile]
    \frametitle{NFSv2协议}
    %    \framesubtitle{xxxx}
    挑战:如何定义无状态文件协议,让它既无状态,又支持POSIX文件系统API?\pause
    \begin{itemize}
        \item 文件句柄(file handle)是有状态的
        \begin{itemize}
            \item 文件句柄用于唯一地描述文件或目录
            \item 许多协议请求包括一个文件句柄        
        \end{itemize} \pause
        \item 服务器启动并再次运行后,客户端会发出第二次读取,但服务器不知道fd指的是哪个文件?
        \item 一个打开文件然后崩溃的客户端:open()在服务器上用掉了一个文件描述符,服务器如何关闭给定的文件呢?
    \end{itemize}

\end{frame}


%----------------------------------------------
\begin{frame}[fragile]
    \frametitle{NFSv2协议}
    %    \framesubtitle{xxxx}
%    挑战:如何定义无状态文件协议,让它既无状态,又支持POSIX文件系统API?
%    \begin{itemize}
%        \item 关键是文件句柄(file handle)。文件句柄用于唯一地描述文件或目录。因此,许多协议请求包括一个文件句柄。        
%        \item 2. 服务器启动并再次运行后,客户端会发出第二次读取,但服务器不知道fd指的是哪个文件?
%        \item 1. 一个打开文件然后崩溃的客户端:open()在服务器上用掉了一个文件描述符,服务器如何关闭给定的文件呢?
%    \end{itemize}
NFSv2协议 part1
    \begin{figure}
        \includegraphics[width=0.9\linewidth]{figs/nfsv2-1.png}
        %  \caption{xxxx}
    \end{figure}
    
\end{frame}
%----------------------------------------------
\begin{frame}[fragile]
    \frametitle{NFSv2协议}
    %    \framesubtitle{xxxx}
    %    挑战:如何定义无状态文件协议,让它既无状态,又支持POSIX文件系统API?
    %    \begin{itemize}
    %        \item 关键是文件句柄(file handle)。文件句柄用于唯一地描述文件或目录。因此,许多协议请求包括一个文件句柄。        
    %        \item 2. 服务器启动并再次运行后,客户端会发出第二次读取,但服务器不知道fd指的是哪个文件?
    %        \item 1. 一个打开文件然后崩溃的客户端:open()在服务器上用掉了一个文件描述符,服务器如何关闭给定的文件呢?
    %    \end{itemize}
    NFSv2协议 part2
    \begin{figure}
        \includegraphics[width=0.8\linewidth]{figs/nfsv2-2.png}
        %  \caption{xxxx}
    \end{figure}
    
\end{frame}


%----------------------------------------------
\begin{frame}[fragile]
    \frametitle{NFSv2协议}
    %    \framesubtitle{xxxx}
    %    挑战:如何定义无状态文件协议,让它既无状态,又支持POSIX文件系统API?
    %    \begin{itemize}
    %        \item 关键是文件句柄(file handle)。文件句柄用于唯一地描述文件或目录。因此,许多协议请求包括一个文件句柄。        
    %        \item 2. 服务器启动并再次运行后,客户端会发出第二次读取,但服务器不知道fd指的是哪个文件?
    %        \item 1. 一个打开文件然后崩溃的客户端:open()在服务器上用掉了一个文件描述符,服务器如何关闭给定的文件呢?
    %    \end{itemize}
    读取文件:客户端和文件服务器的操作 part1
    \begin{figure}
        \includegraphics[width=0.8\linewidth]{figs/nfsv2-read-1.png}
        %  \caption{xxxx}
    \end{figure}
    
\end{frame}

%----------------------------------------------
\begin{frame}[fragile]
    \frametitle{NFSv2协议}
    %    \framesubtitle{xxxx}
    %    挑战:如何定义无状态文件协议,让它既无状态,又支持POSIX文件系统API?
    %    \begin{itemize}
    %        \item 关键是文件句柄(file handle)。文件句柄用于唯一地描述文件或目录。因此,许多协议请求包括一个文件句柄。        
    %        \item 2. 服务器启动并再次运行后,客户端会发出第二次读取,但服务器不知道fd指的是哪个文件?
    %        \item 1. 一个打开文件然后崩溃的客户端:open()在服务器上用掉了一个文件描述符,服务器如何关闭给定的文件呢?
    %    \end{itemize}
    读取文件:客户端和文件服务器的操作 part2
    \begin{figure}
        \includegraphics[width=0.7\linewidth]{figs/nfsv2-read-2.png}
        %  \caption{xxxx}
    \end{figure}
    
\end{frame}

%----------------------------------------------
\begin{frame}
    \frametitle{提纲} % Table of contents slide, comment this block out to remove it
    \tableofcontents % Throughout your presentation, if you choose to use \section{} and \subsection{} commands, these will automatically be printed on this slide as an overview of your presentation
    
    %% itemize
    %Ref:
    %    \begin{itemize}
    %        \item \href{http://osq.cs.berkeley.edu/public/JFoster-Drivers.ppt}{Linux Device Drivers Overview}
    %        \item \href{http://ermak.cs.nstu.ru/understanding.linux.kernel.pdf}{Understanding the Linux Kernel}
    %    \end{itemize}
    
\end{frame}
%----------------------------------------------
    \subsection{NFSv2: 幂等操作}
%----------------------------------------------
\begin{frame}[fragile]
    \frametitle{NFSv2协议:幂等操作}
    %    \framesubtitle{xxxx}
    幂等性

        \begin{itemize}
            \item 如果操作执行多次的效果与执行一次的效果相同,该操作就是幂等的(idempotent)  \pause
            \begin{itemize}
                \item “将值存储到内存中”是幂等的
                \item “将计数器加1”不是幂等的
            \end{itemize} \pause
            \item LOOKUP和READ请求是幂等的
            \begin{itemize}
                \item 只从文件服务器读取信息而不修改
            \end{itemize}
            \item WRITE请求是幂等的
            \begin{itemize}
                \item WRITE消息包含数据、计数和(重要的)写入数据的确切偏移量
                \item 多次写入的结果与单次的结果相同
            \end{itemize}
        \end{itemize}
    
%    \begin{figure}
%        \includegraphics[width=0.5\linewidth]{figs/nfsv2-lose.png}
%        %  \caption{xxxx}
%    \end{figure}
    
\end{frame}

%----------------------------------------------
\begin{frame}[fragile]
    \frametitle{NFSv2协议}
    %    \framesubtitle{xxxx}
    %    挑战:如何定义无状态文件协议,让它既无状态,又支持POSIX文件系统API?
    %    \begin{itemize}
    %        \item 关键是文件句柄(file handle)。文件句柄用于唯一地描述文件或目录。因此,许多协议请求包括一个文件句柄。        
    %        \item 2. 服务器启动并再次运行后,客户端会发出第二次读取,但服务器不知道fd指的是哪个文件?
    %        \item 1. 一个打开文件然后崩溃的客户端:open()在服务器上用掉了一个文件描述符,服务器如何关闭给定的文件呢?
    %    \end{itemize}
    利用幂等操作处理服务器故障 (3种类型的丢失)
    \begin{figure}
        \includegraphics[width=0.5\linewidth]{figs/nfsv2-lose.png}
        %  \caption{xxxx}
    \end{figure}
    
\end{frame}

%----------------------------------------------
\begin{frame}[fragile]
    \frametitle{NFSv2协议}
    %    \framesubtitle{xxxx}
    客户端以统一的幂等操作方式处理了消息丢失和服务器故障
    
    \begin{itemize}
        \item WRITE请求丢失:客户端将重试它,服务器将执行写入,一切都会好
        \item 服务器关闭:第一次请求发送时,服务器恰好关闭;第二次请求发送时,服务器已重启并继续运行,则又会如愿执行
        \item 回复丢失:服务器收到了WRITE请求,发出写入磁盘并发送回复时,回复丢失。这时客户端重新发送请求,服务器再次收到请求时,第二次将数据写入磁盘,并回复它已完成该操作。如果客户端收到第二个回复,则一切正常
    \end{itemize}
    挑战:{\color{blue}一些操作(如mkdir)很难成为幂等的}
    %    \begin{figure}
    %        \includegraphics[width=0.5\linewidth]{figs/nfsv2-lose.png}
    %        %  \caption{xxxx}
    %    \end{figure}
    
\end{frame}

%----------------------------------------------
\begin{frame}
    \frametitle{提纲} % Table of contents slide, comment this block out to remove it
    \tableofcontents % Throughout your presentation, if you choose to use \section{} and \subsection{} commands, these will automatically be printed on this slide as an overview of your presentation
    
    %% itemize
    %Ref:
    %    \begin{itemize}
    %        \item \href{http://osq.cs.berkeley.edu/public/JFoster-Drivers.ppt}{Linux Device Drivers Overview}
    %        \item \href{http://ermak.cs.nstu.ru/understanding.linux.kernel.pdf}{Understanding the Linux Kernel}
    %    \end{itemize}
    
\end{frame}
%----------------------------------------------
    \subsection{NFSv2: 客户端缓存}
%----------------------------------------------
\begin{frame}[fragile]
    \frametitle{NFSv2协议}
    %    \framesubtitle{xxxx}
    提高性能:客户端缓存
    
    \begin{itemize}
        \item NFS客户端文件系统缓存文件数据(和元数据)
        \item 缓存用作写入的临时缓冲区
    \end{itemize}

        \begin{figure}
            \includegraphics[width=0.7\linewidth]{figs/nfsv2-cache.png}
            %  \caption{xxxx}
        \end{figure}
    
\end{frame}

%----------------------------------------------
\begin{frame}[fragile]
    \frametitle{NFSv2协议}
    %    \framesubtitle{xxxx}
    潜在问题:缓存一致性问题
    
    \begin{itemize}
        \item “更新可见性(update visibility)”问题
        \begin{itemize}
            \item  来自一个客户端的更新,什么时候被其他客户端看见?\pause
        \end{itemize}
        \item “陈旧的缓存(stale cache)“问题
        \begin{enumerate}
            \item  C2的写入发送给文件服务器,服务器具有最新版本(F[v2])
            \item  C1缓存中仍然是F[v1]
            \item  C1读文件F,将获得过时的版本(F[v1]),而不是最新的版本(F[v2])
        \end{enumerate}
    \end{itemize}
    
    \begin{figure}
        \includegraphics[width=0.5\linewidth]{figs/nfsv2-cache.png}
        %  \caption{xxxx}
    \end{figure}
    
\end{frame}


%----------------------------------------------
\begin{frame}[fragile]
    \frametitle{关闭时刷新}
    %    \framesubtitle{xxxx}
“关闭时刷新”(flush-on-close,close-to-open)是在客户端实现的一致性语义
    \begin{itemize}
        \item 解决“更新可见性(update visibility)”问题 \pause
        \item 应用程序写入文件并随后关闭文件时,客户端将所有更新(即缓存中的脏页面)刷新到服务器
        \item 通过关闭时刷新的一致性,NFS可确保后续从另一个节点打开文件时,会看到最新的文件版本
    \end{itemize}
    
    \begin{figure}
        \includegraphics[width=0.5\linewidth]{figs/nfsv2-cache.png}
        %  \caption{xxxx}
    \end{figure}
    
\end{frame}

%----------------------------------------------
\begin{frame}[fragile]
    \frametitle{检查文件状态后使用缓存}
    %    \framesubtitle{xxxx}
    NFSv2客户端会先检查文件是否已更改,然后再使用其缓存内容
    \begin{itemize}
        \item 解决“陈旧的缓存(stale cache)”问题 \pause
        \item 打开文件时,客户端文件系统会发出GETATTR请求,以获取文件的属性“服务器上次修改文件信息”
        \item 如果文件修改的时间晚于文件提取到客户端缓存的时间,则客户端会让文件无效(invalidate),从客户端缓存中删除,以确保后续的读取将转向服务器取得该文件的最新版本
    \end{itemize}
    
%    \begin{figure}
%        \includegraphics[width=0.5\linewidth]{figs/nfsv2-cache.png}
%        %  \caption{xxxx}
%    \end{figure}
    
\end{frame}


%----------------------------------------------
\begin{frame}[fragile]
    \frametitle{NFSv2服务器端写缓冲问题}
    %    \framesubtitle{xxxx}
    客户端发出以下写入序列
%    \begin{itemize}
%        \item NFSv2客户端会先检查文件是否已更改,然后再使用其缓存内容。具体来说,在打开文件时,客户端文件系统会发出GETATTR请求,以获取文件的属性。
%        \item 属性包含有关服务器上次修改文件的信息。如果文件修改的时间晚于文件提取到客户端缓存的时间,则客户端会让文件无效(invalidate),因此将它从客户端缓存中删除,并确保后续的读取将转向服务器,取得该文件的最新版本。
%    \end{itemize}
    
        \begin{figure}
            \includegraphics[width=.8\linewidth]{figs/nfsv2-srv1.png}
            %  \caption{xxxx}
        \end{figure}
    
\end{frame}

%----------------------------------------------
\begin{frame}[fragile]
    \frametitle{NFSv2服务器端写缓冲问题}
    %    \framesubtitle{xxxx}
    潜在问题
    \begin{itemize}
        \item 服务器接收到第一个WRITE消息,将它发送到磁盘,并向客户端通知成功
        \item 第二次写入只是缓冲在内存中,服务器在强制写入磁盘之前,向客户端报告成功;但服务器在写入磁盘之前崩溃了
        \item 服务器快速重启,并接收第三个写请求,该请求操作成功
    \end{itemize}
    
    这时,对于客户端,所有请求都成功了,但文件的内容如下:
    %    \begin{itemize}
    %        \item NFSv2客户端会先检查文件是否已更改,然后再使用其缓存内容。具体来说,在打开文件时,客户端文件系统会发出GETATTR请求,以获取文件的属性。
    %        \item 属性包含有关服务器上次修改文件的信息。如果文件修改的时间晚于文件提取到客户端缓存的时间,则客户端会让文件无效(invalidate),因此将它从客户端缓存中删除,并确保后续的读取将转向服务器,取得该文件的最新版本。
    %    \end{itemize}
    
    \begin{figure}
        \includegraphics[width=0.8\linewidth]{figs/nfsv2-srv2.png}
        %  \caption{xxxx}
    \end{figure} \pause
    
    解决方法
    \begin{itemize}
        \item NFS服务器在通知客户端成功之前,将每次写入提交到持久存储
        \item 客户端在写入期间检测到服务器故障后可重试,直到最终成功
    \end{itemize}
\end{frame}

%----------------------------------------------
\begin{frame}[fragile]
    \frametitle{NFSv2协议 -- 小结}
    %    \framesubtitle{xxxx}
    
        \begin{itemize}
            \item NFS的核心
            \begin{itemize}
                \item 简单快速地恢复服务器故障
            \end{itemize}
            \item 幂等文件操作
            \begin{itemize}
                \item 不论服务器是否已执行该请求,客户端可安全地重试失败的操作
            \end{itemize}
            \item 客户端缓存一致性
            \begin{itemize}
                \item 多客户端单服务器NFS系统中,须小心处理缓存一致性问题,才能合理地运行
            \end{itemize}
        \end{itemize}

\end{frame}
%----------------------------------------------
\begin{frame}[fragile]
    \frametitle{MadFS-清华同学用Rust写的分布式文件系统}
    %    \framesubtitle{xxxx}
    \begin{itemize}
        \item \href{https://www.cs.tsinghua.edu.cn/info/1088/4532.htm}{计算机系存储系统研发团队再次刷新超算存储IO500排行榜世界纪录} \pause
        \item 王润基的报告:我眼中的Rust与系统软件(\href{https://lexiangla.com/teams/k100041/classes/6cc584c2f60211eb9714fe9a0679e97d/courses/b72ad2e2f99011eba321c6d7890bac21}{PDF}、\href{https://lexiangla.com/teams/k100041/classes/6cc584c2f60211eb9714fe9a0679e97d/courses/b74aac7af99011eb8f4ac6d7890bac21}{报告视频})
    \end{itemize}
    \begin{figure}
        \includegraphics[width=0.8\linewidth]{figs/MadFS.png}
        %  \caption{xxxx}
    \end{figure}
\end{frame}
%----------------------------------------------
\end{document}
