%%%%%%%%%%%%%%%%%%%%%%%%%%%%%%%%%%%%%%%%%
% Beamer Presentation
% LaTeX Template
% Version 1.0 (10/11/12)
%
% This template has been downloaded from:
% http://www.LaTeXTemplates.com
%
% License:
% CC BY-NC-SA 3.0 (http://creativecommons.org/licenses/by-nc-sa/3.0/)
%
%%%%%%%%%%%%%%%%%%%%%%%%%%%%%%%%%%%%%%%%%

%----------------------------------------------------------------------------------------
%	PACKAGES AND THEMES
%----------------------------------------------------------------------------------------

\documentclass[UTF8,aspectratio=169]{ctexbeamer}

\usepackage{hyperref}
\hypersetup{
	colorlinks=true,
	linkcolor=red,
	anchorcolor=blue,
	citecolor=green
}

\mode<presentation> {
	
	% The Beamer class comes with a number of default slide themes
	% which change the colors and layouts of slides. Below this is a list
	% of all the themes, uncomment each in turn to see what they look like.
	
	%\usetheme{default}
	%\usetheme{AnnArbor}
	%\usetheme{Antibes}
	%\usetheme{Bergen}
	%\usetheme{Berkeley}
	%\usetheme{Berlin}
	%\usetheme{Boadilla}
	%\usetheme{CambridgeUS}
	%\usetheme{Copenhagen}
	%\usetheme{Darmstadt}
	%\usetheme{Dresden}
	%\usetheme{Frankfurt}
	%\usetheme{Goettingen}
	%\usetheme{Hannover}
	%\usetheme{Ilmenau}
	%\usetheme{JuanLesPins}
	%\usetheme{Luebeck}
	\usetheme{Madrid}
	%\usetheme{Malmoe}
	%\usetheme{Marburg}
	%\usetheme{Montpellier}
	%\usetheme{PaloAlto}
	%\usetheme{Pittsburgh}
	%\usetheme{Rochester}
	%\usetheme{Singapore}
	%\usetheme{Szeged}
	%\usetheme{Warsaw}
	
	% As well as themes, the Beamer class has a number of color themes
	% for any slide theme. Uncomment each of these in turn to see how it
	% changes the colors of your current slide theme.
	
	%\usecolortheme{albatross}
	%\usecolortheme{beaver}
	%\usecolortheme{beetle}
	%\usecolortheme{crane}
	%\usecolortheme{dolphin}
	%\usecolortheme{dove}
	%\usecolortheme{fly}
	%\usecolortheme{lily}
	%\usecolortheme{orchid}
	%\usecolortheme{rose}
	%\usecolortheme{seagull}
	%\usecolortheme{seahorse}
	%\usecolortheme{whale}
	%\usecolortheme{wolverine}
	
	%\setbeamertemplate{footline} % To remove the footer line in all slides uncomment this line
	%\setbeamertemplate{footline}[page number] % To replace the footer line in all slides with a simple slide count uncomment this line
	
	%\setbeamertemplate{navigation symbols}{} % To remove the navigation symbols from the bottom of all slides uncomment this line
}

\usepackage{graphicx} % Allows including images
\usepackage{booktabs} % Allows the use of \toprule, \midrule and \bottomrule in tables

% Fonts
% \usepackage{libertine}
% \setmonofont{Courier}
\setCJKsansfont[ItalicFont=Noto Serif CJK SC Black, BoldFont=Noto Sans CJK SC Black]{Noto Sans CJK SC}

%----------------------------------------------------------------------------------------
%	TITLE PAGE
%----------------------------------------------------------------------------------------

\title[第1讲]{第1讲 :操作系统概述} % The short title appears at the bottom of every slide, the full title is only on the title page
\subtitle{第一节:课程概述}
\author{向勇、陈渝} % Your name
\institute[清华大学] % Your institution as it will appear on the bottom of every slide, may be shorthand to save space
{
	清华大学计算机系 \\ % Your institution for the title page
	\medskip
	\textit{xyong,yuchen@tsinghua.edu.cn} % Your email address
}
\date{\today} % Date, can be changed to a custom date


%----------------------------------------------------------------------------------------
%	TITLE PAGE
%----------------------------------------------------------------------------------------

\title[第16讲]{第十八讲 :文件系统实例} % The short title appears at the bottom of every slide, the full title is only on the title page
\subtitle{第1节:FAT文件系统}
\author{向勇、陈渝} % Your name
\institute[清华大学] % Your institution as it will appear on the bottom of every slide, may be shorthand to save space
{
	清华大学计算机系 \\ % Your institution for the title page
	\medskip
	\textit{xyong,yuchen@tsinghua.edu.cn} % Your email address
}
\date{\today} % Date, can be changed to a custom date

\begin{document}

\begin{frame}
\titlepage % Print the title page as the first slide
\end{frame}

%----------------------------------------------
\begin{frame}
\frametitle{提纲} % Table of contents slide, comment this block out to remove it
\tableofcontents % Throughout your presentation, if you choose to use \section{} and \subsection{} commands, these will automatically be printed on this slide as an overview of your presentation

\end{frame}
%----------------------------------------------
%%	PRESENTATION SLIDES
%----------------------------------------------
\section{第1节:FAT文件系统} % Sections can be created in order to organize your presentation into discrete blocks, all sections and subsections are automatically printed in the table of contents as an overview of the talk
%----------------------------------------------
\subsection{xxxx} % A subsection can be created just before a set of slides with a common theme to further break down your presentation into chunks
%----------------------------------------------
\begin{frame}[fragile]
    \frametitle{File Allocation Table (FAT) Volume}
%    \framesubtitle{xxxx}
\end{frame}
%----------------------------------------------
% ###   18.1  File Allocation Table (FAT)
% 
% #### FAT Volume
% 
% ##### File Allocation Table (FAT)
% 
% %% itemize
%     \begin{itemize}
%         \item xx
%     \end{itemize}
% - A simple file system originally **designed for small disks and simple folder structures**. 
% - The FAT file system is named for its method of organization, the **file allocation table**, which resides at the beginning of the volume.
% - To protect the volume, two copies of the table are kept, in case one becomes damaged.
% - **The file allocation tables and the root folder must be stored in a fixed location** so that the files needed to start the system can be correctly located.
% 
%----------------------------------------------
\begin{frame}[fragile]
    \frametitle{xxxx}
%    \framesubtitle{xxxx}
\end{frame}
%----------------------------------------------
% ##### Structure of a FAT Volume
% 
% %% itemize
%     \begin{itemize}
%         \item xx
%     \end{itemize}
% - Boot sector
% - FAT1
% - FAT2
% - Root directory
% - Other directories and all files
% 
% %% figure
% 	\begin{figure}
% 	\includegraphics[width=0.8\linewidth]{test}
% 	\caption{xxxx}
% 	\end{figure}
% ![FAT-volume](figs/FAT-volume.png)
% 
%----------------------------------------------
\begin{frame}[fragile]
    \frametitle{xxxx}
%    \framesubtitle{xxxx}
\end{frame}
%----------------------------------------------
% ##### Differences Between FAT Systems
% 
% %% figure
% 	\begin{figure}
% 	\includegraphics[width=0.8\linewidth]{test}
% 	\caption{xxxx}
% 	\end{figure}
% ![FAT-version](figs/FAT-version.png)
% 
% %% itemize
%     \begin{itemize}
%         \item xx
%     \end{itemize}
% - FAT32 is a derivative of the File Allocation Table (FAT) file system that supports drives with over 2GB of storage.
% - FAT32 drives can contain more than 65,526 clusters and results in more efficient space allocation on the FAT32 drive.
% 
%----------------------------------------------
\subsection{xxxx} % A subsection can be created just before a set of slides with a common theme to further break down your presentation into chunks
%----------------------------------------------
\begin{frame}[fragile]
    \frametitle{File Allocation System}
%    \framesubtitle{xxxx}
\end{frame}
%----------------------------------------------
% #### File Allocation System
% 
% ##### File Allocation System
% 
% The file allocation table contains the following **types** of information about each cluster on the volume (see example below for FAT16):
% %% itemize
%     \begin{itemize}
%         \item xx
%     \end{itemize}
%  * Unused (0x0000)
%  * Cluster in use by a file
%  * Bad cluster (0xFFF7)
%  * Last cluster in a file (0xFFF8-0xFFFF)
% 
%----------------------------------------------
\begin{frame}[fragile]
    \frametitle{xxxx}
%    \framesubtitle{xxxx}
\end{frame}
%----------------------------------------------
% ##### Example of File Allocation Table
% 
% %% figure
% 	\begin{figure}
% 	\includegraphics[width=0.8\linewidth]{test}
% 	\caption{xxxx}
% 	\end{figure}
% ![FAT-example](figs/FAT-example.png)
% 
%----------------------------------------------
\begin{frame}[fragile]
    \frametitle{xxxx}
%    \framesubtitle{xxxx}
\end{frame}
%----------------------------------------------
% ##### FAT Root Folder
% 
% The root folder contains an entry for each file and folder on the root. The only difference between the root folder and other folders is that **the root folder is on a specified location on the disk and has a fixed size** (512 entries for a hard disk, number of entries on a floppy disk depends on the size of the disk).
% 
%----------------------------------------------
\begin{frame}[fragile]
    \frametitle{xxxx}
%    \framesubtitle{xxxx}
\end{frame}
%----------------------------------------------
% ##### Folder Entry
% 
% The Folder Entry includes the following information:
% %% itemize
%     \begin{itemize}
%         \item xx
%     \end{itemize}
%  * Name (eight-plus-three characters)
%  * Attribute byte (8 bits worth of information, described later in this section)
%  * Create time (24 bits)
%  * Create date (16 bits)
%  * Last access date (16 bits)
%  * Last modified time (16 bits)
%  * Last modified date (16 bits.)
%  * Starting cluster number in the file allocation table (16 bits)
%  * File size (32 bits)
% 
%----------------------------------------------
\subsection{xxxx} % A subsection can be created just before a set of slides with a common theme to further break down your presentation into chunks
%----------------------------------------------
\begin{frame}[fragile]
    \frametitle{Filenames on FAT Volumes}
%    \framesubtitle{xxxx}
\end{frame}
%----------------------------------------------
% #### Filenames on FAT Volumes
% 
% ##### Filenames on FAT Volumes
% 
% %% itemize
%     \begin{itemize}
%         \item xx
%     \end{itemize}
% - FAT creates an **eight-plus-three name** for the file. In addition to this conventional entry. 
% - FAT creates **one or more secondary folder entries** for the file, one for each 13 characters in the **long filename**. Each of these secondary folder entries stores a corresponding part of the long filename in Unicode.
% - FAT sets the volume, read-only, system, and hidden **file attribute bits** of the secondary folder entry to mark it as part of a long filename. 
% 
%----------------------------------------------
\begin{frame}[fragile]
    \frametitle{xxxx}
%    \framesubtitle{xxxx}
\end{frame}
%----------------------------------------------
% ##### Folder Entries for the long filename
% 
% Below shows all of the folder entries for the file **Thequi~1.fox**, which has a long name of **The quick brown.fox**. The long name is in Unicode, so each character in the name uses two bytes in the folder entry. The **attribute field** for the long name entries has the value **0x0F**. The attribute field for the short name is **0x20**
% 
%----------------------------------------------
\begin{frame}[fragile]
    \frametitle{xxxx}
%    \framesubtitle{xxxx}
\end{frame}
%----------------------------------------------
% ##### Filenames on FAT Volumes
% 
% %% figure
% 	\begin{figure}
% 	\includegraphics[width=0.8\linewidth]{test}
% 	\caption{xxxx}
% 	\end{figure}
% ![FAT-filename](figs/FAT-filename.png)
%----------------------------------------------
\end{document}
