%%%%%%%%%%%%%%%%%%%%%%%%%%%%%%%%%%%%%%%%%
% Beamer Presentation
% LaTeX Template
% Version 1.0 (10/11/12)
%
% This template has been downloaded from:
% http://www.LaTeXTemplates.com
%
% License:
% CC BY-NC-SA 3.0 (http://creativecommons.org/licenses/by-nc-sa/3.0/)
%
%%%%%%%%%%%%%%%%%%%%%%%%%%%%%%%%%%%%%%%%%

%----------------------------------------------------------------------------------------
%	PACKAGES AND THEMES
%----------------------------------------------------------------------------------------

\documentclass[UTF8,aspectratio=169]{ctexbeamer}

\usepackage{hyperref}
\hypersetup{
	colorlinks=true,
	linkcolor=red,
	anchorcolor=blue,
	citecolor=green
}

\mode<presentation> {
	
	% The Beamer class comes with a number of default slide themes
	% which change the colors and layouts of slides. Below this is a list
	% of all the themes, uncomment each in turn to see what they look like.
	
	%\usetheme{default}
	%\usetheme{AnnArbor}
	%\usetheme{Antibes}
	%\usetheme{Bergen}
	%\usetheme{Berkeley}
	%\usetheme{Berlin}
	%\usetheme{Boadilla}
	%\usetheme{CambridgeUS}
	%\usetheme{Copenhagen}
	%\usetheme{Darmstadt}
	%\usetheme{Dresden}
	%\usetheme{Frankfurt}
	%\usetheme{Goettingen}
	%\usetheme{Hannover}
	%\usetheme{Ilmenau}
	%\usetheme{JuanLesPins}
	%\usetheme{Luebeck}
	\usetheme{Madrid}
	%\usetheme{Malmoe}
	%\usetheme{Marburg}
	%\usetheme{Montpellier}
	%\usetheme{PaloAlto}
	%\usetheme{Pittsburgh}
	%\usetheme{Rochester}
	%\usetheme{Singapore}
	%\usetheme{Szeged}
	%\usetheme{Warsaw}
	
	% As well as themes, the Beamer class has a number of color themes
	% for any slide theme. Uncomment each of these in turn to see how it
	% changes the colors of your current slide theme.
	
	%\usecolortheme{albatross}
	%\usecolortheme{beaver}
	%\usecolortheme{beetle}
	%\usecolortheme{crane}
	%\usecolortheme{dolphin}
	%\usecolortheme{dove}
	%\usecolortheme{fly}
	%\usecolortheme{lily}
	%\usecolortheme{orchid}
	%\usecolortheme{rose}
	%\usecolortheme{seagull}
	%\usecolortheme{seahorse}
	%\usecolortheme{whale}
	%\usecolortheme{wolverine}
	
	%\setbeamertemplate{footline} % To remove the footer line in all slides uncomment this line
	%\setbeamertemplate{footline}[page number] % To replace the footer line in all slides with a simple slide count uncomment this line
	
	%\setbeamertemplate{navigation symbols}{} % To remove the navigation symbols from the bottom of all slides uncomment this line
}

\usepackage{graphicx} % Allows including images
\usepackage{booktabs} % Allows the use of \toprule, \midrule and \bottomrule in tables

% Fonts
% \usepackage{libertine}
% \setmonofont{Courier}
\setCJKsansfont[ItalicFont=Noto Serif CJK SC Black, BoldFont=Noto Sans CJK SC Black]{Noto Sans CJK SC}

%----------------------------------------------------------------------------------------
%	TITLE PAGE
%----------------------------------------------------------------------------------------

\title[第1讲]{第1讲 :操作系统概述} % The short title appears at the bottom of every slide, the full title is only on the title page
\subtitle{第一节:课程概述}
\author{向勇、陈渝} % Your name
\institute[清华大学] % Your institution as it will appear on the bottom of every slide, may be shorthand to save space
{
	清华大学计算机系 \\ % Your institution for the title page
	\medskip
	\textit{xyong,yuchen@tsinghua.edu.cn} % Your email address
}
\date{\today} % Date, can be changed to a custom date


%----------------------------------------------------------------------------------------
%	TITLE PAGE
%----------------------------------------------------------------------------------------

\title[第12讲]{第十二讲:多处理器调度} % The short title appears at the bottom of every slide, the full title is only on the title page
\subtitle{第三节:O(1) 调度}
\author{向勇、陈渝、李国良} % Your name
\institute[清华大学] % Your institution as it will appear on the bottom of every slide, may be shorthand to save space
{
	清华大学计算机系 \\ % Your institution for the title page
	\medskip
	\textit{xyong,yuchen,liguoliang@tsinghua.edu.cn} % Your email address
}
\date{\today} % Date, can be changed to a custom date


\begin{document}

\begin{frame}
\titlepage % Print the title page as the first slide
\end{frame}

%\begin{frame}
%\frametitle{提纲} % Table of contents slide, comment this block out to remove it
%\tableofcontents % Throughout your presentation, if you choose to use \section{} and \subsection{} commands, these will automatically be printed on this slide as an overview of your presentation
%\end{frame}
%
%%----------------------------------------------------------------------------------------
%%	PRESENTATION SLIDES\begin{itemize}
%%----------------------------------------------------------------------------------------
\subsection{Linux内核}
%----------------------------------------------
\begin{frame}
\frametitle{提纲} % Table of contents slide, comment this block out to remove it
\tableofcontents % Throughout your presentation, if you choose to use \section{} and \subsection{} commands, these will automatically be printed on this slide as an overview of your presentation

\end{frame}
%----------------------------------------------
% 谈谈调度 - Linux O(1)  https://cloud.tencent.com/developer/article/1077507
% Linux O(1) CPU Scheduler http://people.cs.ksu.edu/~gud/docs/ppt/scheduler.pdf
\begin{frame}
	\frametitle{SMP 和 Linux 内核}
	\begin{columns}
	\begin{column}{.4\textwidth}
	\Large \centering
	
    \includegraphics[width=1.\textwidth]{single-queue}
	\includegraphics[width=1.\textwidth]{sqms}	
	\end{column}
	
	\begin{column}{.6\textwidth}
%		\large
\begin{itemize}
	\item 在 Linux 2.0 的早期,SMP 支持由一个 “大锁” 组成,这个 “大锁” 对操作系统内部的访问进行串行化
	\item 在2.2前的内核中,SMP实现在用户级,Linux内核本身并不能因为有多个处理器而得到加速

	\end{itemize}

	\end{column}
\end{columns}
\end{frame}


\begin{frame}
	\frametitle{Linux 2.4 内核:SMP实现在核心级}
	
	\begin{itemize}
		
		\item  使用多处理器可以加快内核的处理速度,调度器是复杂度为O(n)
		
		\begin{itemize}
			\item 内核调度器维护两个 queue:runqueue 和 expired queue
			\item 两个 queue 都永远保持有序
			\item 一个 process 用完时间片,就会被插入 expired queue
			\item 当 runqueue 为空时,只需要把 runqueue 和 expired queue 交换一下即可
		\end{itemize}
	\end{itemize}
	
    \begin{figure}
        \includegraphics[width=0.8\textwidth,natwidth=1011,natheight=343]{figs/linux-2.4-sched.png}
    \end{figure}

\end{frame}

\begin{frame}
	\frametitle{Linux 2.4 内核:SMP实现在核心级}
	
	\begin{itemize}
		
		\item  使用多处理器可以加快内核的处理速度,调度器是复杂度为O(n)
	
		\begin{itemize}
		\item 全局共享的就绪队列
		\item 寻找下一个可执行的 process,这个操作一般都是 O(1)
		\item 每次进程用完时间片,找合适的位置执行插入操作,会遍历所有任务,复杂度为O(n)

		\end{itemize}
	\end{itemize}
	
    \begin{figure}
	\includegraphics[width=0.8\textwidth,natwidth=1011,natheight=343]{figs/linux-2.4-sched.png}
    \end{figure}


\end{frame}

\begin{frame}
	\frametitle{Linux 2.4 内核:SMP实现在核心级}
	
	\begin{itemize}
		
		\item  使用多处理器可以加快内核的处理速度,调度器是复杂度为O(n)
		
		\begin{itemize}
			\item 现代操作系统都能运行成千上万个进程
			\item O(n) 算法意味着每次调度时,对于当前执行完的 process,需要把所有在 expired queue 中的 process 过一遍,找到合适的位置插入
			\item 这不仅仅会带来性能上的巨大损失,还使得系统的调度时间非常不确定 —— 根据系统的负载,可能有数倍甚至数百倍的差异
		\end{itemize}
	\end{itemize}
	
    \begin{figure}
    \includegraphics[width=0.8\textwidth,natwidth=1011,natheight=343]{figs/linux-2.4-sched.png}
    \end{figure}
	
	
\end{frame}
%%------------------------------------------------
\subsection{满足O(1)的数据结构}
%----------------------------------------------
\begin{frame}
\frametitle{提纲} % Table of contents slide, comment this block out to remove it
\tableofcontents % Throughout your presentation, if you choose to use \section{} and \subsection{} commands, these will automatically be printed on this slide as an overview of your presentation

\end{frame}
%----------------------------------------------
\begin{frame}
	\frametitle{ O(1) 调度器}
	\begin{columns}
		\begin{column}{.5\textwidth}

            \begin{figure}
    			\includegraphics[width=1.\textwidth]{O1}
            \end{figure}
	
		\end{column}
		
		\begin{column}{.5\textwidth}
			\large
			O(1) 调度器 \\
			
2.6 版本的调度器是由 Ingo Molnar 设计并实现的。Ingo 从 1995 年开始就一直参与 Linux 内核的开发。他编写这个新调度器的动机是为唤醒、上下文切换和定时器中断开销建立一个完全 O(1) 的调度器
			
		\end{column}
	\end{columns}
\end{frame}


%%------------------------------------------------
\begin{frame}
	\frametitle{O(1)数据结构:随机访问}
	\begin{columns}
		\begin{column}{.4\textwidth}
            \begin{figure}
    			\includegraphics[width=1.\textwidth]{O1}
            \end{figure}
			
		\end{column}
		
		\begin{column}{.6\textwidth}
%			\large
			满足 O(1) 的数据结构?\\
			回顾一下数据结构的四种基本操作和时间复杂度
			\begin{itemize}
			\item access:随机访问
				\begin{itemize}
				\item array: 平均情况和最坏情况均能达到 O(1)
				\item linked list 是 O(N)
				\item tree 一般是 O(log N)
				\end{itemize}
			\end{itemize}
		\end{column}
	\end{columns}
\end{frame}


%%------------------------------------------------
\begin{frame}
	\frametitle{O(1)数据结构:搜索}
	\begin{columns}
		\begin{column}{.4\textwidth}
	\includegraphics[width=1.\textwidth]{O1}
			
		\end{column}
		
		\begin{column}{.6\textwidth}
			%			\large
			满足 O(1) 的数据结构?\\
			回顾一下数据结构的四种基本操作和时间复杂度
			\begin{itemize}
				\item search:搜索
				\begin{itemize}
					\item hash table 时间复杂度是O(1),但它最坏情况下是 O(N) 
					\item 大部分 tree(b-tree / red-black tree)平均和最坏情况都是 O(log N)
				\end{itemize}
			\end{itemize}
		\end{column}
	\end{columns}
\end{frame}


%%------------------------------------------------
\begin{frame}
	\frametitle{O(1)数据结构:插入和删除}
	\begin{columns}
		\begin{column}{.4\textwidth}
			\includegraphics[width=1.\textwidth]{O1}
			
		\end{column}
		
		\begin{column}{.6\textwidth}
			%			\large
			满足 O(1) 的数据结构?\\
			回顾一下数据结构的四种基本操作和时间复杂度
			\begin{itemize}
				\item insert/deletion:插入和删除
				\begin{itemize}
					\item hash table 时间复杂度是O(1),但它最坏情况下是 O(N) 
					\item linked list,stack,queue 在平均和最坏情况下都是 O(1)
				\end{itemize}
			\end{itemize}
		\end{column}
	\end{columns}
\end{frame}
%%------------------------------------------------
\subsection{O(1)调度器}
%----------------------------------------------
\begin{frame}
\frametitle{提纲} % Table of contents slide, comment this block out to remove it
\tableofcontents % Throughout your presentation, if you choose to use \section{} and \subsection{} commands, these will automatically be printed on this slide as an overview of your presentation

\end{frame}
%----------------------------------------------
\begin{frame}
	\frametitle{ O(1) 调度器}
	\begin{columns}
		\begin{column}{.6\textwidth}
	\includegraphics[width=1.\textwidth]{O1-2}
		\end{column}
		
		\begin{column}{.4\textwidth}
			%			\large

			\begin{itemize}
				\item 进程有140种优先级,可用长度为 140 的 array 去记录优先级。access是O(1)

				\item 每个优先级下面用一个 FIFO queue 管理这个优先级下的 process。新来的插到队尾,先进先出,insert / deletion 都是 O(1)

			\end{itemize}
		\end{column}
	\end{columns}
\end{frame}

%%------------------------------------------------
\begin{frame}
	\frametitle{ O(1) 调度器}
	\begin{columns}
		\begin{column}{.4\textwidth}
			\includegraphics[width=1.\textwidth]{O1-2}
			
		\end{column}
		
		\begin{column}{.6\textwidth}

			\begin{itemize}
				\item 进程有140种优先级,可用长度为 140 的 array 去记录优先级。access是O(1)
				
					\begin{itemize}
					\item bitarray,它为每种优先级分配一个 bit,如果这个优先级队列下面有 process,那么就对相应的 bit 染色,置为 1,否则置为 0。
					
					\item 问题可简化成寻找一个 bitarray 里面最高位是 1 的 bit(left-most bit),这可用一条 CPU 指令实现。
%					BSR BSF  指令 Intel 80386 https://docs.oracle.com/cd/E19120-01/open.solaris/817-5477/eoizi/index.html
					\end{itemize}
				
			\end{itemize}
		\end{column}
	\end{columns}
\end{frame}

%%------------------------------------------------
\begin{frame}
	\frametitle{ O(1) 调度器}
	\begin{columns}
		\begin{column}{.4\textwidth}
			\includegraphics[width=1.\textwidth]{O1-2}
			
		\end{column}
		
		\begin{column}{.6\textwidth}

			\begin{itemize}
				\item 在active bitarray(APA)中寻找 left-most bit 的位置 x。
			\item 在APA中找到对应队列 APA[x]。
			\item 从 APA[x] 中 dequeue 一个 process。
			\item 对于当前执行完的 process,重新计算其 priority,然后 enqueue 到 expired priority array(EPA)相应的队里 EPA[priority]。
			\item 如果 priority 在 expired bitarray 里对应的 bit 为 0,将其置 1。
			\item 如果 active bitarray 全为零,将 active bitarray 和 expired bitarray 交换一下。
			\end{itemize}
		\end{column}
	\end{columns}
\end{frame}



%%------------------------------------------------
\begin{frame}
	\frametitle{O(1)调度器的多核支持}
	\begin{columns}
		\begin{column}{.3\textwidth}
			\includegraphics[width=1.\textwidth]{O1-2}
			
		\end{column}
		
		\begin{column}{.7\textwidth}
			%			\large
			% O(1) 调度器:多核/SMP支持\\
			
			\begin{itemize}
				\item 在一定时间间隔后,进行load balance分析
				\item  rq­> cpu\_load : represents load on the CPU
				
				\item 在每个时钟中断后进行计算
				\item \small current\_load = rq­>nr\_running * SCHED\_LOAD\_SCALE;
				
				\item 由负载轻的CPU pulling 进程而不是 pushing进程
				
			\end{itemize}
		\end{column}
	\end{columns}
\end{frame}
%----------------------------------------------
%----------------------------------------------
%----------------------------------------------
\end{document}
